\section{Conclusion}

The motivation of this research was offering an approach to the $p$-hub problem using GA. Implementing the 
\code{USApHMP-Q} function as the basis of the fitness function and with tailored \emph{crossover} and \emph{mutation}.

In overall terms the implemented algorithm fulfilled its objective as heuristic. With an execution below
the single second, the algorithm can generate a solution with a median gap of just $2.255\%$ over the best 
solution; ranging from half a dozen of optimal solutions found and maximum $9.619\%$ gap. With our analysis
we found the mutation probability of $0.5$ being the best performant of those tested (with a deeper analysis
we could find a better candidate in its vicinity), using that value the solution would rise to a $2.066\%$
median gap and a $8.628\%$ maximum gap.

Regarding the fit of these implementation to find the optimal solution the algorithm lacks quite a bit.
The GA accomplishes good results for the easiest subproblems, but fails to find the solution with the upper
set of subproblems. The cause of this is a set of possible solutions that outgrowns the capabilities of the
implementation. Several enchancements could be added to improve the GA in this aspect, like implementing a
selection phase evaluating more candidates, a new mutation creating more variance or a replacement that could
keep some sub-optimal but highly different solutions that could open up new paths during the crossover step.

All in all, these improvements could not be enough to close the gap enough with other approaches used in the
literature with this problem. This limitation is closely related with the foundations of the genetic algorithms
and would end up being a stone in the road sooner or later.


