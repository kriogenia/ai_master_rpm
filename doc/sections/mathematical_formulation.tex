\section{Mathematical formulation}

The problem \code{USApHMP-Q} is the quadratic linear algorithm of O'Kelly (1997)\cite{OKelly1987}
reformulated by Ernst \& Krishnamoorthy (1996)\cite{Ernst1996} to accomodate the new costs
introduced on their problem. Let Z be a matrix \emph{nxn} indicating which nodes of \emph{N} are
allocated to which hubs. The total cost of a certain allocation \emph{Z} is the sum of the
delivery cost every path \emph{i,j}, being the delivery cost of a path the product of the
flow between the nodes and the sum of each step of the delivery.

\break

\textbf{Notation:}

\wij: flow between the nodes \emph{i} and \emph{j}

\point{d}{ij}: distance between the nodes \emph{i} and \emph{j} 

$\chi$: collection cost

$\alpha$: hub-to-hub transport cost

$\delta$: distribution cost
\[
Z_{ij} = \begin{cases}
  1 & \text{if } i \text{ is assigned to hub } j, \forall i = 1,\dots,n, \forall j = 1,\dots,n \\
  0 & \text{otherwise, }\forall i = 1,\dots,n, \forall j = 1,\dots,n \\
\end{cases}
\]

\subsection{Problem \code{USAp-HMP-Q}}

\[
  \text{Min.} \quad \sum_{i\in N}\sum_{j\in N}\sum_{k\in N}\sum_{l\in N} W_{ij}(\chi d_{ik} Z_{ik} + \alpha d_{kl} Z_{ik} Z_{jl} + \delta d_{jl} Z_{jl})
\]

\begin{alignat}{3}
  \text{S.t.} \quad & \sum_{k \in N} & \quad Z_{kk} &= p         & \label{p_hubs}\\
                    & \sum_{k \in N} & \quad Z_{kk} &= 1         & \quad & \forall i \in N \label{single_alloc} \\
                    &                & \quad Z_{ik} &\le Z_{kk}  & \quad & \forall i, k \in N \label{only_hubs} \\
                    &                & \quad Z_{ik} &\in \{0,1\} & \quad & \forall i, k \in N \label{prev_realloc}
\end{alignat}

In this problem \eqref{p_hubs} sets the \ip-hub part of the problem
ensuring that there's only \ip hubs. \eqref{single_alloc} ensures the single
allocation as each column of \emph{Z} can only sum 1. \eqref{only_hubs} prevents
allocations to non-hub nodes. And \eqref{prev_realloc} asserts that hub
nodes are not allocated to other nodes.

Aside from this formulation, Ernst \& Krishnamoorthy present two other versions in the
same paper\cite{Ernst1996}. To see other formulations around the same or similar problems
refer to the survey conducted by Farahani (2013)\cite{Farahani2013}.

\subsection{Fitness function}

The fitness function used in this project uses the \code{USAp-HMP-Q} function as its basis.
In order to transform it into a maximization function it's negated. As we'll be using as a
chromosome a valid and complete solution (see \ref{chromosome}) 
where each node is already allocated to one of \emph{p} nodes the equation can be greatly
simplified as there's no need for the two summatories over \emph{k} and \emph{l} to find the
allocated nodes of \emph{i} and \emph{j}. Letting \point{S}{i} $\in$ \{1,\dots,n\} $\forall$ i be an
array of size \emph{n} where \point{S}{i} is the assigned hub of \emph{i}, the fitness
function to maximize is as follows:


\[
  \text{Max.} \quad 1 - \sum_{i \in n} \sum_{j \in n} W_{ij} (\chi d_{iS_{i}} + \alpha d_{S_{i}S_{j}} + \delta d_{jS_{j}})
\]

\begin{alignat}{3}
  \text{S.t.} \quad & |U(S)|  &= p \label{ensure_p}
\end{alignat}

In this case aside from the possible values or the length or \emph{S} the only requirement
the function must subject to is that the total of uniques values of \emph{S} is \emph{p}
as it's enforced with Equation (\ref{ensure_p}).

% todo think possible further restrictions?


