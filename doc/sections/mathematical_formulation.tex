\section{Mathematical formulation}

The problem \code{USApHMP-Q} is the quadratic linear algorithm of O'Kelly (1997)\cite{OKelly1987}
reformulated by Ernst \& Krishnamoorthy (1996)\cite{Ernst1996} to accomodate the new costs
introduced on their problem. Let $Z$ be a matrix $nxn$ indicating which nodes of $N$ are allocated
to which hubs. The total cost of a certain allocation $Z$ is the sum of the delivery cost every
path $i,j$, being the delivery cost of a path the product of the flow between the nodes and the 
sum of each step of the delivery.

\hfill

\textbf{Notation:}

$W_{ij}$: flow between the nodes $i$ and $j$

$d_{ij}$: euclidean distance between the nodes $i$ and $j$

$\chi$: collection cost

$\alpha$: hub-to-hub transport cost

$\delta$: distribution cost
\[
Z_{ij} = \begin{cases}
  1 & \text{if } i \text{ is assigned to hub } j, \forall i = 1,\dots,n, \forall j = 1,\dots,n \\
  0 & \text{otherwise, }\forall i = 1,\dots,n, \forall j = 1,\dots,n \\
\end{cases}
\]

\subsection{Problem \code{USAp-HMP-Q}}

\[
  \text{Min.} \quad \sum_{i\in N}\sum_{j\in N}\sum_{k\in N}\sum_{l\in N} W_{ij}(\chi d_{ik} Z_{ik} + \alpha d_{kl} Z_{ik} Z_{jl} + \delta d_{jl} Z_{jl})
\]

\begin{alignat}{3}
  \text{S.t.} \quad & \sum_{k \in N} & \quad Z_{kk} &= p         & \label{p_hubs}\\
                    & \sum_{k \in N} & \quad Z_{kk} &= 1         & \quad & \forall i \in N \label{eq:single_alloc} \\
                    &                & \quad Z_{ik} &\le Z_{kk}  & \quad & \forall i, k \in N \label{eq:only_hubs} \\
                    &                & \quad Z_{ik} &\in \{0,1\} & \quad & \forall i, k \in N \label{eq:prev_alloc}
\end{alignat}

In this problem \eqref{p_hubs} sets the $p$-hub part of the problem ensuring that there's
only $p$ hubs. \eqref{eq:single_alloc} ensures the single allocation as  each column of
$Z$ can only sum 1. \eqref{eq:only_hubs} prevents allocations to non-hub nodes. And
\eqref{eq:prev_alloc} asserts that hub nodes are not allocated to other nodes.

Aside from this formulation, Ernst \& Krishnamoorthy present two other versions in the
same paper\cite{Ernst1996}. To see other formulations around the same or similar problems
refer to the survey conducted by Farahani (2013)\cite{Farahani2013}.

\subsection{Fitness function}

The fitness function used in this project uses the \code{USAp-HMP-Q} function as its basis.
In order to transform it into a maximization function it's negated. As we'll be using as a
chromosome a valid and complete solution (see \ref{ss:chromosome}) 
where each node is already allocated to one of $p$ nodes the equation can be greatly simplified
as there's no need for the two summatories over $k$ and $l$ to find the allocated nodes of $i$
and $j$. Letting $S_{i} \in \{1,\dots,n\} \forall i$ be an array of size $n$ where $S_{i}$
is the assigned hub of $i$, the fitness function to maximize is:

\[
  \text{Max.} \quad 1 - \sum_{i \in n} \sum_{j \in n} W_{ij} (\chi d_{iS_{i}} + \alpha d_{S_{i}S_{j}} + \delta d_{jS_{j}})
\]

\begin{alignat}{3}
  \text{S.t.} \quad & |S| = n \\
                    & S_{i} \in \{1,\dots,n\}, \forall i \in \{1,\dots,n\} \\
   & |U(S)| = p \label{eq:ensure_p}
\end{alignat}

In this case aside from the possible values or the length or $S$ the only requirement
the function must subject to is that the total of uniques values of $S$ is $p$
as it's enforced with Equation (\ref{eq:ensure_p}).
