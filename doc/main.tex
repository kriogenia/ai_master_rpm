\documentclass[onecolumn]{IEEEtran}

\usepackage{algorithm,algpseudocode}
\usepackage{amsmath}
\usepackage{array}
\usepackage{authblk}
\usepackage{caption}
\usepackage[utf8]{inputenc}
\usepackage{mathtools}
\usepackage{nameref}
\usepackage{graphicx}
% \graphicspath{ {images/} }

% metadata
\author{Ricardo Soto Estévez}
\affil{Menendez Pelayo International University}
\date{\today}
\title{Practical study of basic Genetic Algorithm over the p-Hub Median Problem}

% custom macros
\newcommand{\ip}{\emph{p}} % todo remove
\newcommand{\wij}{\emph{W\textsubscript{ij}}}
\def\code#1{\texttt{#1}}
\def\point#1#2{\emph{#1\textsubscript{#2}}}

\begin{document}

\maketitle

\begin{abstract} \noindent {
This article addresses the single allocation p-hub median problem by applying a genetic algorithm (GA)
to optimize hub placements. It details the fitness function for evaluating candidate solutions and describes
the genetic operators used to generate new populations. Performance benchmarks across various problem sizes
are presented to compare implementations and identify the most effective configurations.
} \end{abstract}

\begin{IEEEkeywords}
Hub location, genetic algorithms, heuristic solution, metaheuristics, analysis 
\end{IEEEkeywords}

\section{Introduction}

The hub location problem was first introduced by O'Kelly (1987)\cite{OKelly1987}. It stems from real world
industries, like postal deliveries or passengers transports, where different places are nodes in an
interconnected graph and all possible transits should be optimized defining some of them as hubs. These
centrals hubs would serve as switching points for flows between the nodes, optimizing the edges connecting
two hubs. Each node would be assigned to a hub and the transit from a node \emph{i} to a node \emph{j}
would be routed first to the hub assigned to ii, \emph{k}, and then to \emph{j} via its own assigned hub, 
\emph{l}. This structure creates a network where the positioning of hubs significantly 
impacts transportation costs. The hub location model seeks to minimize these costs by choosing optimal hub
placements.

The first approach proposed by O'Kelly (1987)\cite{OKelly1987} was a quadratic integer program with a
non-convex objective function, what easily points this as a NP-hard problem with high computation costs
and a general magnitude of \emph{O(n\textsuperscript(4))}. This sparkled several articles in the literature
with new approaches and variants to this problem, see: Campbell (1996)\cite{Campbell1996}, O'Kelly (1992)
\cite{OKelly1992} and Aykin (1994)\cite{Aykin1994}.

From this subsequent articles on the matter we are the most interested in those of A.T. Ernst \&
M. Krishnamoorthy\cite{Ernst1996}\cite{Ernst1998}\cite{Ernst1999}. In these papers the authors expand
the problem defining three different costs for each type of connection (spoke-to-hub, hub-to-hub and
hub-to-spoke) instead of the uniform cost or the alpha reduction that was used in the previous
enunciates. On this research we'll tackle the problem specified in "\emph{Efficient algorithms for
the uncapacitated single allocation \ip-hub median problem}"\cite{Ernst1996} version, but the three
papers work over the same dataset with different restrictions.

This papers studies the uncapacitated single allocation \ip-hub median problem (from now on
\code{USApHMP}). The scenario of this problem is a complete graph \emph{G=(N,E)} where \emph{N={1, ..., n}} is
the set of nodes and \emph{E = N x N} is the set of edges connecting every node with the rest.
The volume of traffic between two nodes \emph{i} and \emph{j} is given in the flows matrix \emph{W}, 
being \wij the flow demand from \emph{i} to \emph{j} (in this variant of the problem this
is not necessarily symmetrical).

This problem is \emph{uncapacitated} as the hubs are assumed to be able to handle an unlimited
capacity of transfers. It's \emph{single} as each node can only have one assigned hub. And
it's a \ip-hub problem as \ip is the fixed numbers of hubs that must emplaced. To see a version
with a variable \ip or capacitated hubs, please consult \cite{Ernst1999}. For a version with 
allowing multiple allocations, refer to \cite{Ernst1998}.

The main difference of this variant with previous literatures is the presence of three separate
costs associated with each flow \wij. The \emph{collection} ($\chi$) cost is associated to the allocation 
of the transfered item from the origin node to its assigned hub; the \emph{transfer} ($\alpha$) cost is 
applied to the movement between hubs; and the \emph{distribution} ($\delta$) refers to the remaining step, 
from the last hub to the terminal location. Each one of this is proportional to the distance between 
the connected nodes.

These three papers of Ernst and Krishnamoorthy use a dataset from the \emph{Australia Post} with 200
nodes and the respective 200x200 flows matrix. A C program was attached to the dataset to reduce
it into smaller problems given a certain N and P. The optimal solutions for every combination of 
\emph{N = {10, 20, 30, 40, 50}} and \emph{P = {2, 3, 4, 5}} were also provided. In this paper we'll
be working with this dataset and these smaller problems.

Following this introduction we'll... 

% TODO add structure summary



\input{sections/mathematical_formulation.tex}

\section{Genetic Algorithm}

Our implementation follows the basic schema of every GA\cite{Thede2004}. Creation and evaluation of an initial
population (\emph{P}), and improvement of that population until a termination condition is met. The population
refinement  is done via a \emph{selection} of two or more parents, generation of new childs from the
\emph{crossover} of these parents, application of a \emph{mutation} to diversify the population and a
evaluation of the childs to select and \emph{refine} the new population that will be used in the next
iteration of the loop.

% todo name the vars like population_size? <-- assign a greek letter and modify the population defininition

In order to improve the performance of the algorithm some domain-specific components were implemented,
mainly in the \nameref{ss:crossover} (\ref{ss:crossover}) 
and \nameref{ss:mutation} (\ref{ss:mutation}) steps. This 
section will offer a brief description of the implementation of every facet of the GA.

\subsection{Chromosome encoding\label{ss:chromosome}}

The chromosome encoding that we used in our individuals is an array \emph{S} of size \emph{n} where each
index would point to the randomly assigned hub of each node. On top of that, these individuals are built
with the necessary contracts to ensure that they always contain \emph{p} unique values stored in the chromosome,
that way we are ensuring that every chromosome is a valid solution with \emph{p} hubs.

\begin{equation}
  S = [1, 1, 5, 1, 5, 5, 9, 1, 5, 9 ],\quad p = 3 \label{eq:chromosome_example}
\end{equation}

An example of this chromosome to a problem of size \emph{n = 10} and \emph{p = 3} is the one shown
at (\ref{eq:chromosome_example}). In this example the node 1 is assigned to itself as indicated
by $S_{1}=1$, the node 2 is allocated also to node 1, the node 3 to another hub placed in 5 and so on.
See also that $U(S)=\{1,5,9\}$ fulfilling the $|U(S)|=3$ requirement to have $p=3$ hubs.

The generation of the initial population of individuals is random, selecting three different values from
${1,\dots,n}$ and randomly assigning them to the different hubs. To ensure that all three have at least
one assigned hub, these first random chromosome will always allocated each hub to itself (\ref{eq:autoalloc}).

\begin{equation}
  S_{i}=i, \forall i \in U(S) \label{eq:autoalloc}
\end{equation}

\subsection{Selection}

In the selection step the strategy used is the \emph{Binary Tournament}. We pick two random individuals from
the population and select the one with the higher fitness. In case of equal fitness the first candidate
gets the priority. Both random individuals can be the same, in that case, that individual is ensured to
be picked for the crossover step.

This is done twice to select a total of two parents.  Both selected parents can also be the same individual,
in that case the child generated will be a clone of the parent, but can still provide a better solution with
the mutation.

\subsection{Crossover\label{ss:crossover}}

The crossover applied in the algorithm is based on the \emph{Single Point Crossover} strategy, but adding
two possible modifications to ensure the validity of the solutions, as the merged chromosome can
have between $1$ and $2p$ nodes, as it is shown in (\ref{eq:bad_chromosome}) with two parents and a
cutting point that can generate two invalid chromosomes, the first one with a number of hubs below
p and the second one with almost twice the value of $p$.

\begin{equation}
  \label{eq:bad_chromosome}
  [1,1,1,1,2,3] x [5,1,5,4,1,1] \xRightarrow{x=4} [1,1,1,1,1,1], [5,1,5,4,2,3]
\end{equation}

This single point version generates a cutting point $x \in \{1,\dots,n\}$, creates a new child, and
then replicates the hub of every node $i$ from the left parent while $i<=x$ and the right
parent for every $x<i<=n$. If $x=n$ then the child will be an exact replica of the left parent.

Then it evaluates the resulting child looking at its number of hubs. If it's exactly $p$ the child
is deemed valid and returned. If it's below $p$, new hubs are generated until $p$ is satisfied, reallocating
each of these nodes converted to hubs to themselves. If it's above $p$, random hubs are selected and
removed until the child matches $p$, every orphaned node from this purge is rellocated to one of the
remaining nodes. See the algorithm \ref{alg:crossover} for the full breakdown.

\begin{algorithm}[H]
  \caption{Crossover}
  \label{alg:crossover}
  \begin{algorithmic}[1]
    \Require{$L$ (left parent)}
    \Require{$R$ (right parent)}
    \Require{$p$}
    \Ensure{valid child}
    \Function{RemoveHub}{$S,p$}
      \While{$length(S.hubs)>p$}
        \State $x \gets randomNode(S)$
        \State $y \gets randomNode(S - nodeToRemove)$
        \For{$n$ in $0, length(S)$}
          \If {$S_{n} = x$}
            \State $S_{n} \gets y$
          \EndIf
        \EndFor
      \EndWhile
    \EndFunction
    \item[]
    \Function{AddHub}{$S,p$}
      \While{$length(S.hubs)<p$}
        \State $x \gets randomNode(S - S.hubs)$
        \State $S_{x} \gets x$
      \EndWhile
    \EndFunction
  \item[]
  \State $x \gets randomPoint(N)$
  \State $C \gets L.replica()$
  \For{$i$ in ($x$, $length(child)$)}
    \State $C_{i} \gets R_{i}$
  \EndFor
  \If{$length(C.hubs) > p$}
    $RemoveHub(C,p)$
  \ElsIf{$length(C.hubs) < p$}
    $AddHub(C,p)$
  \EndIf
  \State \Return C
  \end{algorithmic}
\end{algorithm}

\subsection{Mutation\label{ss:mutation}}

The mutation applied to the chromosome is the a hub reassignment. A random non-hub node is
picked and transformed into a hub. Every sibling node assigned to the same hub is reallocated
to this new hub, including the previous hub (see Alg. \ref{alg:mutation}).
The probability to perform this mutation can be specified before running the algorithm,
and different values ($\{0.05., 0.1, 0.25, 0.5\}$) were tested in the following section 
(\ref{s:computational_results})
to find which probability performed the best.

\begin{algorithm}[H]
  \caption{Mutation}
  \label{alg:mutation}
  \begin{algorithmic}[1]
    \Require{$S$ (chromosome)}
    \State $h \gets randomNode(S - S.hubs)$
    \For{$n$ in $0, length(S)$}
    \If {$S_{n} = S_{h}$}
        \State $S_{n} \gets h$
      \EndIf
    \EndFor
  \end{algorithmic}
\end{algorithm}

\subsection{Replacement}

To perform the replacement that generates the population of the next interaction a simple
elitist approach was followed. Adding the new child to the current population in the place
of the current worst individual if it has a better fitness.

\section{Computational results\label{s:computational_results}}

% TODO
%
% 30 evals

\subsection{Performance in a fixed number of evaluations\label{ss:benchmark_evaluations}}

In this analysis, the algorithm was executed with a fixed number of maximum evaluations: $100000$
for each $(n,p)$ problem and \emph{mutation probability} combination. The population for each
problem was set to $n$. The fitness of the best solution, the gap between this fitness and the best
solution, and the consumed time was recorded for each of this executions.

The aim of this benchmark is to showcase the overall results for the different mutation probabilities.
This way we can discern the best value or range to specify in the GA when it's executed in a restricted
timeframe and the optimal is not required.  Table \ref{tb:evals_benchmark}
contains the data for the solutions up until $n=40$. The data containing the solutions for size $50$
can be found in the \code{data/benchmark\_evals\_2024-11-14T09-27.csv} file at the repository\cite{SotoEstevezGA2024}.

\input{tables/benchmark_evaluations.tex}

From this data two aggregations were conducted to analyze the behavior of \emph{mutation} for each
different problem size. The two aggregations were groupings by $(p,mutation)$ and $(n,mutation)$.
The Figure \ref{fig:mutation_vs_gap}
shows a comparison between $n$ and $p$ against the resulting mean $gap$ with the four different mutations.

\begin{figure}[h!]
  \centering
  \includegraphics[width=0.9\textwidth]{figures/mutation_vs_gap.png}
  \caption{p vs. Gap and n vs. Gap based on the mutation probability}
  \label{fig:mutation_vs_gap}
\end{figure}

From this result we can see that overall the highest mutation probability ($0.5$) gets the results
closer to the optimal solution, except on the problems were either $n$ or $p$ are minimal. In these
cases the minimum mutation ($0.05$) comes on top for $n=10$ and all the mutation values get almost
the same gap for $p=2$. Outside of these minimal problems were the set of solutions is not big
enough to get an advantage from a diverse population, the lower mutation probability performs
significantly worse, specially as $n$ and $p$ increases.

\subsection{Performance searching the optimal solution\label{ss:benchmark_optimal}}

During this benchmark the algorithm was executed until the solution given by the examples
was matched. A hard limit of one million evaluations was set to evade 

Refer to Table \ref{tb:optimal_benchmark}.

\begin{table}
\caption{Results of optimal search benchmark}
\label{tb:optimal_benchmark}
\begin{center}
  \begin{tabular}{|c|c|c|r|r|c|r|r|r|}
\hline
n & p & mutation & objective & fitness & gap (\%) & hits (\%) & evaluations & time (ms) \\
\hline
10 & 2 & 0.05 & 167493 & 167493 &  0.0000 & 83.3 & 1374 & 8.585 \\
10 & 2 & 0.10 & 167493 & 167493 &  0.0000 & 93.3 & 389 & 0.787 \\
10 & 2 & 0.25 & 167493 & 167493 &  0.0000 & 96.7 & 244 & 0.529 \\
10 & 2 & 0.50 & 167493 & 167493 &  0.0000 & 90.0 & 1813 & 3.780 \\
10 & 3 & 0.05 & 136008 & 136008 &  0.0000 & 46.7 & 48413 & 81.867 \\
10 & 3 & 0.10 & 136008 & 136008 &  0.0000 & 70.0 & 30050 & 50.900 \\
10 & 3 & 0.25 & 136008 & 136008 &  0.0000 & 56.7 & 2178 & 4.122 \\
10 & 3 & 0.50 & 136008 & 136008 &  0.0000 & 70.0 & 4556 & 9.259 \\
10 & 4 & 0.05 & 112396 & 112396 &  0.0000 & 33.3 & 20309 & 34.350 \\
10 & 4 & 0.10 & 112396 & 112396 &  0.0000 & 36.7 & 3516 & 6.255 \\
10 & 4 & 0.25 & 112396 & 112396 &  0.0000 & 43.3 & 6907 & 12.847 \\
10 & 4 & 0.50 & 112396 & 112396 &  0.0000 & 36.7 & 6033 & 12.327 \\
10 & 5 & 0.05 & 91105 & 91105 &  0.0000 & 66.7 & 39487 & 65.813 \\
10 & 5 & 0.10 & 91105 & 91105 &  0.0000 & 60.0 & 44018 & 77.774 \\
10 & 5 & 0.25 & 91105 & 91105 &  0.0000 & 63.3 & 22759 & 41.629 \\
10 & 5 & 0.50 & 91105 & 91105 &  0.0000 & 60.0 & 16208 & 32.840 \\
20 & 2 & 0.05 & 172817 & 172817 &  0.0000 & 100.0 & 1143 & 4.230 \\
20 & 2 & 0.10 & 172817 & 172817 &  0.0000 & 100.0 & 580 & 2.256 \\
20 & 2 & 0.25 & 172817 & 172817 &  0.0000 & 100.0 & 495 & 2.001 \\
20 & 2 & 0.50 & 172817 & 172817 &  0.0000 & 100.0 & 491 & 2.205 \\
20 & 3 & 0.05 & 151533 & 151533 &  0.0000 & 10.0 & 234160 & 826.083 \\
20 & 3 & 0.10 & 151533 & 151548 &  0.0096 & 0.0 & 1000000 & 3429.551 \\
20 & 3 & 0.25 & 151533 & 151533 &  0.0000 & 3.3 & 28299 & 104.198 \\
20 & 3 & 0.50 & 151533 & 151548 &  0.0096 & 0.0 & 1000000 & 3906.435 \\
20 & 4 & 0.05 & 135625 & 135625 &  0.0000 & 20.0 & 10273 & 36.399 \\
20 & 4 & 0.10 & 135625 & 135625 &  0.0000 & 13.3 & 28774 & 103.815 \\
20 & 4 & 0.25 & 135625 & 135625 &  0.0000 & 10.0 & 14737 & 56.926 \\
20 & 4 & 0.50 & 135625 & 135625 &  0.0000 & 16.7 & 32448 & 134.209 \\
20 & 5 & 0.05 & 123130 & 123348 &  0.1764 & 0.0 & 1000000 & 3416.257 \\
20 & 5 & 0.10 & 123130 & 123348 &  0.1764 & 0.0 & 1000000 & 3454.870 \\
20 & 5 & 0.25 & 123130 & 123348 &  0.1764 & 0.0 & 1000000 & 3574.185 \\
20 & 5 & 0.50 & 123130 & 123348 &  0.1764 & 0.0 & 1000000 & 3888.024 \\
25 & 2 & 0.05 & 175542 & 175542 &  0.0000 & 76.7 & 2482 & 12.001 \\
25 & 2 & 0.10 & 175542 & 175542 &  0.0000 & 76.7 & 2621 & 12.422 \\
25 & 2 & 0.25 & 175542 & 175542 &  0.0000 & 90.0 & 2065 & 10.371 \\
25 & 2 & 0.50 & 175542 & 175542 &  0.0000 & 83.3 & 2576 & 13.737 \\
25 & 3 & 0.05 & 155256 & 155375 &  0.0762 & 0.0 & 1000000 & 4192.340 \\
25 & 3 & 0.10 & 155256 & 155256 &  0.0000 & 6.7 & 250119 & 1113.705 \\
25 & 3 & 0.25 & 155256 & 155256 &  0.0000 & 6.7 & 4054 & 20.220 \\
25 & 3 & 0.50 & 155256 & 155256 &  0.0000 & 3.3 & 248197 & 1260.082 \\
25 & 4 & 0.05 & 139197 & 139197 &  0.0000 & 3.3 & 11881 & 55.496 \\
25 & 4 & 0.10 & 139197 & 139197 &  0.0000 & 6.7 & 9238 & 43.410 \\
25 & 4 & 0.25 & 139197 & 139197 &  0.0000 & 6.7 & 49050 & 236.929 \\
25 & 4 & 0.50 & 139197 & 139197 &  0.0000 & 6.7 & 31903 & 169.742 \\
25 & 5 & 0.05 & 123574 & 125028 &  1.1626 & 0.0 & 1000000 & 4354.754 \\
25 & 5 & 0.10 & 123574 & 124961 &  1.1100 & 0.0 & 1000000 & 4450.238 \\
25 & 5 & 0.25 & 123574 & 123574 &  0.0000 & 3.3 & 34009 & 161.110 \\
25 & 5 & 0.50 & 123574 & 123574 &  0.0000 & 3.3 & 39328 & 202.600 \\
\hline
\end{tabular}
\end{center}
\end{table}



% todo name size of the examples

\section{Conclusion}

% maybe the mutation algorithm didnt cause enough variance

\bibliographystyle{IEEEtran}
\bibliography{references}

\end{document}
