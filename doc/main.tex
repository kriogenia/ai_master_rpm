\documentclass[onecolumn]{IEEEtran}

\usepackage{amsmath}
\usepackage{authblk}
\usepackage[utf8]{inputenc}
\usepackage{nameref}
\usepackage{graphicx}
% \graphicspath{ {images/} }

% metadata
\author{Ricardo Soto Estévez}
\affil{Menendez Pelayo International University}
\date{\today}
\title{Practical study of basic Genetic Algorithm over the p-Hub Median Problem}

% custom macros
\newcommand{\ip}{\emph{p}} % todo remove
\newcommand{\wij}{\emph{W\textsubscript{ij}}}
\def\code#1{\texttt{#1}}
\def\point#1#2{\emph{#1\textsubscript{#2}}}

\begin{document}

\maketitle

\begin{abstract} \noindent {
This article addresses the single allocation p-hub median problem by applying a genetic algorithm (GA)
to optimize hub placements. It details the fitness function for evaluating candidate solutions and describes
the genetic operators used to generate new populations. Performance benchmarks across various problem sizes
are presented to compare implementations and identify the most effective configurations.
} \end{abstract}

\begin{IEEEkeywords}
Hub location, genetic algorithms, heuristic solution, metaheuristics, analysis 
\end{IEEEkeywords}

\section{Introduction}

The hub location problem was first introduced by O'Kelly (1987)\cite{OKelly1987}. It stems from real world
industries, like postal deliveries or passenger transports, where different places are nodes in an
interconnected graph and all possible transits should be optimized defining some of them as hubs. These
centrals hubs would serve as switching points for flows between the nodes, optimizing the edges connecting
two hubs. Each node would be assigned to a hub and the transit from a node $i$ to a node $j$
would be routed first to the hub assigned to $i$ ($k$) and then to $j$ via its own assigned hub,
$l$. This structure creates a network where the positioning of hubs significantly impacts transportation
costs. The hub location model seeks to minimize these costs by choosing optimal hub placements.

The first approach proposed by O'Kelly (1987)\cite{OKelly1987} was a quadratic integer program with a
non-convex objective function, what easily points this as a NP-hard problem with high computation costs
and a general magnitude of $O(n^{4})$. This sparkled several articles in the literature
with new approaches and variants to this problem, see: Campbell (1996)\cite{Campbell1996}, O'Kelly (1992)
\cite{OKelly1992} and Aykin (1994)\cite{Aykin1994}.

From this subsequent articles on the matter, we are the most interested in those of A.T. Ernst \&
M. Krishnamoorthy\cite{Ernst1996}\cite{Ernst1998}\cite{Ernst1999}. In these papers the authors expand
the problem defining three different costs for each type of connection (spoke-to-hub, hub-to-hub and
hub-to-spoke) instead of the uniform cost or the alpha reduction that was used in the previous
enunciates. On this research we'll tackle the problem specified in "\emph{Efficient algorithms for
the uncapacitated single allocation $p$-hub median problem}"\cite{Ernst1996} version, but the three
papers work over the same dataset with different restrictions.

These papers study the uncapacitated single allocation $p$-hub median problem (from now on
\code{USApHMP}). The scenario of this problem is a complete graph $G=(N,E)$ where $N={1, ..., n}$ is
the set of nodes and $E = N x N$ is the set of edges connecting every node with the rest.
The volume of traffic between two nodes $i$ and $j$ is given in the flows matrix $W$, being $W_{ij}$ 
the flow demand from $i$ to $j$ (in this variant of the problem this is not necessarily symmetrical).

This problem is \emph{uncapacitated} as the hubs are assumed to be able to handle an unlimited
capacity of transfers. It's \emph{single} as each node can only have one assigned hub. And
it's a $p$-hub problem as $p$ is the fixed numbers of hubs that must be emplaced. To see a version
with a variable $p$ or capacitated hubs, please consult \cite{Ernst1999}. For a version
allowing multiple allocations, refer to \cite{Ernst1998}.

The main difference of this variant with previous literatures is the presence of three separate
costs associated with each flow $W_{ij}$. The \emph{collection} ($\chi$) cost is associated to the allocation
of the transfered item from the origin node to its assigned hub; the \emph{transfer} ($\alpha$) cost is
applied to the movement between hubs; and the \emph{distribution} ($\delta$) refers to the remaining step,
from the last hub to the terminal location. Each one of this is proportional to the distance between
the connected nodes.

These three papers of Ernst and Krishnamoorthy use a dataset from the \emph{Australia Post} with 200
nodes and the respective $200x200$ flows matrix. A C program was attached to the dataset to reduce
it into smaller problems given a certain N and P. The optimal solutions for every combination of
$N = \{10, 20, 30, 40, 50\}$ and $P = \{2, 3, 4, 5\}$ were also provided. In this paper we'll
be working with this dataset and these smaller problems.

Following this introduction we'll see the mathematical formulation for our fitness function, then
we'll see the implementations of the different genetic operators, and after a review of the
computational results we'll extract some conclusiones on this research.

\section{Mathematical formulation}

The problem \code{USApHMP-Q} is the quadratic linear algorithm of O'Kelly (1997)\cite{OKelly1987}
reformulated by Ernst \& Krishnamoorthy (1996)\cite{Ernst1996} to accomodate the new costs
introduced on their problem. Let Z be a matrix \emph{nxn} indicating which nodes of \emph{N} are
allocated to which hubs. The total cost of a certain allocation \emph{Z} is the sum of the
delivery cost every path \emph{i,j}, being the delivery cost of a path the product of the
flow between the nodes and the sum of each step of the delivery.

\break

\textbf{Notation:}

\wij: flow between the nodes \emph{i} and \emph{j}

\point{d}{ij}: distance between the nodes \emph{i} and \emph{j} 

$\chi$: collection cost

$\alpha$: hub-to-hub transport cost

$\delta$: distribution cost
\[
Z_{ij} = \begin{cases}
  1 & \text{if } i \text{ is assigned to hub } j, \forall i = 1,\dots,n, \forall j = 1,\dots,n \\
  0 & \text{otherwise, }\forall i = 1,\dots,n, \forall j = 1,\dots,n \\
\end{cases}
\]

\subsection{Problem \code{USAp-HMP-Q}}

\[
  \text{Min.} \quad \sum_{i\in N}\sum_{j\in N}\sum_{k\in N}\sum_{l\in N} W_{ij}(\chi d_{ik} Z_{ik} + \alpha d_{kl} Z_{ik} Z_{jl} + \delta d_{jl} Z_{jl})
\]

\begin{alignat}{3}
  \text{S.t.} \quad & \sum_{k \in N} & \quad Z_{kk} &= p         & \label{p_hubs}\\
                    & \sum_{k \in N} & \quad Z_{kk} &= 1         & \quad & \forall i \in N \label{single_alloc} \\
                    &                & \quad Z_{ik} &\le Z_{kk}  & \quad & \forall i, k \in N \label{only_hubs} \\
                    &                & \quad Z_{ik} &\in \{0,1\} & \quad & \forall i, k \in N \label{prev_realloc}
\end{alignat}

In this problem \eqref{p_hubs} sets the \ip-hub part of the problem
ensuring that there's only \ip hubs. \eqref{single_alloc} ensures the single
allocation as each column of \emph{Z} can only sum 1. \eqref{only_hubs} prevents
allocations to non-hub nodes. And \eqref{prev_realloc} asserts that hub
nodes are not allocated to other nodes.

Aside from this formulation, Ernst \& Krishnamoorthy present two other versions in the
same paper\cite{Ernst1996}. To see other formulations around the same or similar problems
refer to the survey conducted by Farahani (2013)\cite{Farahani2013}.

\subsection{Fitness function}

The fitness function used in this project uses the \code{USAp-HMP-Q} function as its basis.
In order to transform it into a maximization function it's negated. As we'll be using as a
chromosome a valid and complete solution (see \ref{chromosome}) 
where each node is already allocated to one of \emph{p} nodes the equation can be greatly
simplified as there's no need for the two summatories over \emph{k} and \emph{l} to find the
allocated nodes of \emph{i} and \emph{j}. Letting \point{S}{i} $\in$ \{1,\dots,n\} $\forall$ i be an
array of size \emph{n} where \point{S}{i} is the assigned hub of \emph{i}, the fitness
function to maximize is as follows:


\[
  \text{Max.} \quad 1 - \sum_{i \in n} \sum_{j \in n} W_{ij} (\chi d_{iS_{i}} + \alpha d_{S_{i}S_{j}} + \delta d_{jS_{j}})
\]

\begin{alignat}{3}
  \text{S.t.} \quad & |U(S)|  &= p \label{ensure_p}
\end{alignat}

In this case aside from the possible values or the length or \emph{S} the only requirement
the function must subject to is that the total of uniques values of \emph{S} is \emph{p}
as it's enforced with Equation (\ref{ensure_p}).

% todo think possible further restrictions?




\section{Genetic Algorithm}

Our implementation follows the basic schema of every GA\cite{Thede2004}. Creation and evaluation of an initial
population, and improvement of that population until a termination condition is met. The population refinement 
is done via a \emph{selection} of two or more parents, generation of new childs from the \emph{crossover} of
these parents, application of a \emph{mutation} to diversify the population and a evaluation of the childs
to select and \emph{refine} the new population that will be used in the next iteration of the loop.

In order to improve the performance of the algorithm some domain-specific components were implemented,
mainly in the \nameref{ss:crossover} (\ref{ss:crossover}) 
and \nameref{ss:mutation} (\ref{ss:mutation}) steps. This 
section will offer a brief description of the implementation of every facet of the GA.

\subsection{Chromosome encoding\label{ss:chromosome}}

The chromosome encoding that we used in our individuals is an array \emph{S} of size \emph{n} where each
index would point to the randomly assigned hub of each node. On top of that, these individuals are built
with the necessary contracts to ensure that they always contain \emph{p} unique values stored in the chromosome,
that way we are ensuring that every chromosome is a valid solution with \emph{p} hubs.

\begin{equation}
  S = [1, 1, 5, 1, 5, 5, 9, 1, 5, 9 ],\quad p = 3 \label{eq:chromosome_example}
\end{equation}

An example of this chromosome to a problem of size \emph{n = 10} and \emph{p = 3} is the one shown
at (\ref{eq:chromosome_example}). In this example the node 1 is assigned to itself as indicated
by $S_{1}=1$, the node 2 is allocated also to node 1, the node 3 to another hub placed in 5 and so on.
See also that $U(S)=\{1,5,9\}$ fulfilling the $|U(S)|=3$ requirement to have $p=3$ hubs.

The generation of the initial population of individuals is random, selecting three different values from
${1,\dots,n}$ and randomly assigning them to the different hubs. To ensure that all three have at least
one assigned hub, these first random chromosome will always allocated each hub to itself (\ref{eq:autoalloc}).

\begin{equation}
  S_{i}=i, \forall i \in U(S) \label{eq:autoalloc}
\end{equation}

\subsection{Selection}

\subsection{Crossover\label{ss:crossover}}

\subsection{Mutation\label{ss:mutation}}

\subsection{Replacement}

\section{Computational results}

\section{Conclusion}

\bibliographystyle{IEEEtran}
\bibliography{references}

\end{document}
